\documentclass[12pt, a4paper]{article}

% --- PACOTES BÁSICOS ---
\usepackage[utf8]{inputenc}
\usepackage[T1]{fontenc}
\usepackage[brazil]{babel}
\usepackage{geometry}
\usepackage{amsmath}

% --- CONFIGURAÇÕES DE LAYOUT ---
\geometry{a4paper, margin=1in} % Define margens de 1 polegada (aprox. 2.54cm)
\setlength{\parindent}{0pt} % Remove a indentação do primeiro parágrafo
\setlength{\parskip}{1.5ex} % Adiciona um pequeno espaço entre os parágrafos

% --- INFORMAÇÕES DO DOCUMENTO ---
\title{Respostas -- Atividade de Laboratório: Árvores Binárias}
\author{Gabriel}
\date{\today}


% --- INÍCIO DO DOCUMENTO ---
\begin{document}

\maketitle % Cria o título com as informações acima
\thispagestyle{empty} % Remove o número da página da primeira folha
\vspace{2cm} % Adiciona um espaço vertical

% -----------------------------------------------------------------------------
\section*{Pergunta 1: O que muda na estrutura da árvore quando inserimos os valores em ordem crescente?}
% -----------------------------------------------------------------------------

Absolutamente \textbf{nada} muda na estrutura (no formato) da árvore.

O método de inserção implementado, \texttt{insert\_level\_order}, preenche a árvore estritamente por nível, da esquerda para a direita, ocupando o primeiro espaço que encontrar. Esse algoritmo \textbf{ignora completamente o valor} do dado que está sendo inserido; sua lógica se baseia apenas na topologia da árvore.

Isso significa que a sequência de inserção `[1, 2, 3, 4, 5, 6]` resulta em uma árvore com uma estrutura \textbf{idêntica} à que seria gerada pelas sequências `[6, 5, 4, 3, 2, 1]` ou `[4, 1, 5, 6, 2, 3]`. O que muda são os valores armazenados em cada nó, mas o "desenho" da árvore, suas conexões e sua classificação estrutural permanecem os mesmos.

\begin{itemize}
    \item[\textbf{Nota:}] Este comportamento é fundamentalmente diferente do observado em uma \textbf{Árvore Binária de Busca (BST)}, onde o valor de cada nó é crucial para determinar sua posição, e a ordem de inserção altera drasticamente a estrutura final da árvore.
\end{itemize}


% -----------------------------------------------------------------------------
\section*{Pergunta 2: Por que a árvore resultante não é perfeita?}
% -----------------------------------------------------------------------------

A árvore resultante da inserção dos valores `[1, 2, 3, 4, 5, 6]` não é perfeita porque a definição de uma \textbf{árvore perfeita} exige que \textbf{todos os seus níveis estejam completamente preenchidos}.

Ao analisar a estrutura da árvore gerada, observamos o seguinte preenchimento dos níveis:
\begin{itemize}
    \item \textbf{Nível 0:} Contém o nó 1. (Capacidade: $2^0=1$, Ocupado: 1. \textbf{OK})
    \item \textbf{Nível 1:} Contém os nós 2 e 3. (Capacidade: $2^1=2$, Ocupado: 2. \textbf{OK})
    \item \textbf{Nível 2:} Contém os nós 4, 5 e 6. (Capacidade: $2^2=4$, Ocupado: 3. \textbf{INCOMPLETO})
\end{itemize}

Como o último nível (Nível 2) não está com sua capacidade máxima preenchida, a condição para que a árvore seja perfeita não é satisfeita.

Apesar de não ser perfeita, a árvore é classificada como \textbf{completa}, pois todos os níveis, exceto o último, estão cheios, e os nós do último nível estão alocados o mais à esquerda possível.

\end{document}
